
% Default to the notebook output style

    


% Inherit from the specified cell style.




    
\documentclass[11pt]{article}

    
    
    \usepackage[T1]{fontenc}
    % Nicer default font (+ math font) than Computer Modern for most use cases
    \usepackage{mathpazo}

    % Basic figure setup, for now with no caption control since it's done
    % automatically by Pandoc (which extracts ![](path) syntax from Markdown).
    \usepackage{graphicx}
    % We will generate all images so they have a width \maxwidth. This means
    % that they will get their normal width if they fit onto the page, but
    % are scaled down if they would overflow the margins.
    \makeatletter
    \def\maxwidth{\ifdim\Gin@nat@width>\linewidth\linewidth
    \else\Gin@nat@width\fi}
    \makeatother
    \let\Oldincludegraphics\includegraphics
    % Set max figure width to be 80% of text width, for now hardcoded.
    \renewcommand{\includegraphics}[1]{\Oldincludegraphics[width=.8\maxwidth]{#1}}
    % Ensure that by default, figures have no caption (until we provide a
    % proper Figure object with a Caption API and a way to capture that
    % in the conversion process - todo).
    \usepackage{caption}
    \DeclareCaptionLabelFormat{nolabel}{}
    \captionsetup{labelformat=nolabel}

    \usepackage{adjustbox} % Used to constrain images to a maximum size 
    \usepackage{xcolor} % Allow colors to be defined
    \usepackage{enumerate} % Needed for markdown enumerations to work
    \usepackage{geometry} % Used to adjust the document margins
    \usepackage{amsmath} % Equations
    \usepackage{amssymb} % Equations
    \usepackage{textcomp} % defines textquotesingle
    % Hack from http://tex.stackexchange.com/a/47451/13684:
    \AtBeginDocument{%
        \def\PYZsq{\textquotesingle}% Upright quotes in Pygmentized code
    }
    \usepackage{upquote} % Upright quotes for verbatim code
    \usepackage{eurosym} % defines \euro
    \usepackage[mathletters]{ucs} % Extended unicode (utf-8) support
    \usepackage[utf8x]{inputenc} % Allow utf-8 characters in the tex document
    \usepackage{fancyvrb} % verbatim replacement that allows latex
    \usepackage{grffile} % extends the file name processing of package graphics 
                         % to support a larger range 
    % The hyperref package gives us a pdf with properly built
    % internal navigation ('pdf bookmarks' for the table of contents,
    % internal cross-reference links, web links for URLs, etc.)
    \usepackage{hyperref}
    \usepackage{longtable} % longtable support required by pandoc >1.10
    \usepackage{booktabs}  % table support for pandoc > 1.12.2
    \usepackage[inline]{enumitem} % IRkernel/repr support (it uses the enumerate* environment)
    \usepackage[normalem]{ulem} % ulem is needed to support strikethroughs (\sout)
                                % normalem makes italics be italics, not underlines
    

    
    
    % Colors for the hyperref package
    \definecolor{urlcolor}{rgb}{0,.145,.698}
    \definecolor{linkcolor}{rgb}{.71,0.21,0.01}
    \definecolor{citecolor}{rgb}{.12,.54,.11}

    % ANSI colors
    \definecolor{ansi-black}{HTML}{3E424D}
    \definecolor{ansi-black-intense}{HTML}{282C36}
    \definecolor{ansi-red}{HTML}{E75C58}
    \definecolor{ansi-red-intense}{HTML}{B22B31}
    \definecolor{ansi-green}{HTML}{00A250}
    \definecolor{ansi-green-intense}{HTML}{007427}
    \definecolor{ansi-yellow}{HTML}{DDB62B}
    \definecolor{ansi-yellow-intense}{HTML}{B27D12}
    \definecolor{ansi-blue}{HTML}{208FFB}
    \definecolor{ansi-blue-intense}{HTML}{0065CA}
    \definecolor{ansi-magenta}{HTML}{D160C4}
    \definecolor{ansi-magenta-intense}{HTML}{A03196}
    \definecolor{ansi-cyan}{HTML}{60C6C8}
    \definecolor{ansi-cyan-intense}{HTML}{258F8F}
    \definecolor{ansi-white}{HTML}{C5C1B4}
    \definecolor{ansi-white-intense}{HTML}{A1A6B2}

    % commands and environments needed by pandoc snippets
    % extracted from the output of `pandoc -s`
    \providecommand{\tightlist}{%
      \setlength{\itemsep}{0pt}\setlength{\parskip}{0pt}}
    \DefineVerbatimEnvironment{Highlighting}{Verbatim}{commandchars=\\\{\}}
    % Add ',fontsize=\small' for more characters per line
    \newenvironment{Shaded}{}{}
    \newcommand{\KeywordTok}[1]{\textcolor[rgb]{0.00,0.44,0.13}{\textbf{{#1}}}}
    \newcommand{\DataTypeTok}[1]{\textcolor[rgb]{0.56,0.13,0.00}{{#1}}}
    \newcommand{\DecValTok}[1]{\textcolor[rgb]{0.25,0.63,0.44}{{#1}}}
    \newcommand{\BaseNTok}[1]{\textcolor[rgb]{0.25,0.63,0.44}{{#1}}}
    \newcommand{\FloatTok}[1]{\textcolor[rgb]{0.25,0.63,0.44}{{#1}}}
    \newcommand{\CharTok}[1]{\textcolor[rgb]{0.25,0.44,0.63}{{#1}}}
    \newcommand{\StringTok}[1]{\textcolor[rgb]{0.25,0.44,0.63}{{#1}}}
    \newcommand{\CommentTok}[1]{\textcolor[rgb]{0.38,0.63,0.69}{\textit{{#1}}}}
    \newcommand{\OtherTok}[1]{\textcolor[rgb]{0.00,0.44,0.13}{{#1}}}
    \newcommand{\AlertTok}[1]{\textcolor[rgb]{1.00,0.00,0.00}{\textbf{{#1}}}}
    \newcommand{\FunctionTok}[1]{\textcolor[rgb]{0.02,0.16,0.49}{{#1}}}
    \newcommand{\RegionMarkerTok}[1]{{#1}}
    \newcommand{\ErrorTok}[1]{\textcolor[rgb]{1.00,0.00,0.00}{\textbf{{#1}}}}
    \newcommand{\NormalTok}[1]{{#1}}
    
    % Additional commands for more recent versions of Pandoc
    \newcommand{\ConstantTok}[1]{\textcolor[rgb]{0.53,0.00,0.00}{{#1}}}
    \newcommand{\SpecialCharTok}[1]{\textcolor[rgb]{0.25,0.44,0.63}{{#1}}}
    \newcommand{\VerbatimStringTok}[1]{\textcolor[rgb]{0.25,0.44,0.63}{{#1}}}
    \newcommand{\SpecialStringTok}[1]{\textcolor[rgb]{0.73,0.40,0.53}{{#1}}}
    \newcommand{\ImportTok}[1]{{#1}}
    \newcommand{\DocumentationTok}[1]{\textcolor[rgb]{0.73,0.13,0.13}{\textit{{#1}}}}
    \newcommand{\AnnotationTok}[1]{\textcolor[rgb]{0.38,0.63,0.69}{\textbf{\textit{{#1}}}}}
    \newcommand{\CommentVarTok}[1]{\textcolor[rgb]{0.38,0.63,0.69}{\textbf{\textit{{#1}}}}}
    \newcommand{\VariableTok}[1]{\textcolor[rgb]{0.10,0.09,0.49}{{#1}}}
    \newcommand{\ControlFlowTok}[1]{\textcolor[rgb]{0.00,0.44,0.13}{\textbf{{#1}}}}
    \newcommand{\OperatorTok}[1]{\textcolor[rgb]{0.40,0.40,0.40}{{#1}}}
    \newcommand{\BuiltInTok}[1]{{#1}}
    \newcommand{\ExtensionTok}[1]{{#1}}
    \newcommand{\PreprocessorTok}[1]{\textcolor[rgb]{0.74,0.48,0.00}{{#1}}}
    \newcommand{\AttributeTok}[1]{\textcolor[rgb]{0.49,0.56,0.16}{{#1}}}
    \newcommand{\InformationTok}[1]{\textcolor[rgb]{0.38,0.63,0.69}{\textbf{\textit{{#1}}}}}
    \newcommand{\WarningTok}[1]{\textcolor[rgb]{0.38,0.63,0.69}{\textbf{\textit{{#1}}}}}
    
    
    % Define a nice break command that doesn't care if a line doesn't already
    % exist.
    \def\br{\hspace*{\fill} \\* }
    % Math Jax compatability definitions
    \def\gt{>}
    \def\lt{<}
    % Document parameters
    \title{1-getting\_started}
    
    
    

    % Pygments definitions
    
\makeatletter
\def\PY@reset{\let\PY@it=\relax \let\PY@bf=\relax%
    \let\PY@ul=\relax \let\PY@tc=\relax%
    \let\PY@bc=\relax \let\PY@ff=\relax}
\def\PY@tok#1{\csname PY@tok@#1\endcsname}
\def\PY@toks#1+{\ifx\relax#1\empty\else%
    \PY@tok{#1}\expandafter\PY@toks\fi}
\def\PY@do#1{\PY@bc{\PY@tc{\PY@ul{%
    \PY@it{\PY@bf{\PY@ff{#1}}}}}}}
\def\PY#1#2{\PY@reset\PY@toks#1+\relax+\PY@do{#2}}

\expandafter\def\csname PY@tok@w\endcsname{\def\PY@tc##1{\textcolor[rgb]{0.73,0.73,0.73}{##1}}}
\expandafter\def\csname PY@tok@c\endcsname{\let\PY@it=\textit\def\PY@tc##1{\textcolor[rgb]{0.25,0.50,0.50}{##1}}}
\expandafter\def\csname PY@tok@cp\endcsname{\def\PY@tc##1{\textcolor[rgb]{0.74,0.48,0.00}{##1}}}
\expandafter\def\csname PY@tok@k\endcsname{\let\PY@bf=\textbf\def\PY@tc##1{\textcolor[rgb]{0.00,0.50,0.00}{##1}}}
\expandafter\def\csname PY@tok@kp\endcsname{\def\PY@tc##1{\textcolor[rgb]{0.00,0.50,0.00}{##1}}}
\expandafter\def\csname PY@tok@kt\endcsname{\def\PY@tc##1{\textcolor[rgb]{0.69,0.00,0.25}{##1}}}
\expandafter\def\csname PY@tok@o\endcsname{\def\PY@tc##1{\textcolor[rgb]{0.40,0.40,0.40}{##1}}}
\expandafter\def\csname PY@tok@ow\endcsname{\let\PY@bf=\textbf\def\PY@tc##1{\textcolor[rgb]{0.67,0.13,1.00}{##1}}}
\expandafter\def\csname PY@tok@nb\endcsname{\def\PY@tc##1{\textcolor[rgb]{0.00,0.50,0.00}{##1}}}
\expandafter\def\csname PY@tok@nf\endcsname{\def\PY@tc##1{\textcolor[rgb]{0.00,0.00,1.00}{##1}}}
\expandafter\def\csname PY@tok@nc\endcsname{\let\PY@bf=\textbf\def\PY@tc##1{\textcolor[rgb]{0.00,0.00,1.00}{##1}}}
\expandafter\def\csname PY@tok@nn\endcsname{\let\PY@bf=\textbf\def\PY@tc##1{\textcolor[rgb]{0.00,0.00,1.00}{##1}}}
\expandafter\def\csname PY@tok@ne\endcsname{\let\PY@bf=\textbf\def\PY@tc##1{\textcolor[rgb]{0.82,0.25,0.23}{##1}}}
\expandafter\def\csname PY@tok@nv\endcsname{\def\PY@tc##1{\textcolor[rgb]{0.10,0.09,0.49}{##1}}}
\expandafter\def\csname PY@tok@no\endcsname{\def\PY@tc##1{\textcolor[rgb]{0.53,0.00,0.00}{##1}}}
\expandafter\def\csname PY@tok@nl\endcsname{\def\PY@tc##1{\textcolor[rgb]{0.63,0.63,0.00}{##1}}}
\expandafter\def\csname PY@tok@ni\endcsname{\let\PY@bf=\textbf\def\PY@tc##1{\textcolor[rgb]{0.60,0.60,0.60}{##1}}}
\expandafter\def\csname PY@tok@na\endcsname{\def\PY@tc##1{\textcolor[rgb]{0.49,0.56,0.16}{##1}}}
\expandafter\def\csname PY@tok@nt\endcsname{\let\PY@bf=\textbf\def\PY@tc##1{\textcolor[rgb]{0.00,0.50,0.00}{##1}}}
\expandafter\def\csname PY@tok@nd\endcsname{\def\PY@tc##1{\textcolor[rgb]{0.67,0.13,1.00}{##1}}}
\expandafter\def\csname PY@tok@s\endcsname{\def\PY@tc##1{\textcolor[rgb]{0.73,0.13,0.13}{##1}}}
\expandafter\def\csname PY@tok@sd\endcsname{\let\PY@it=\textit\def\PY@tc##1{\textcolor[rgb]{0.73,0.13,0.13}{##1}}}
\expandafter\def\csname PY@tok@si\endcsname{\let\PY@bf=\textbf\def\PY@tc##1{\textcolor[rgb]{0.73,0.40,0.53}{##1}}}
\expandafter\def\csname PY@tok@se\endcsname{\let\PY@bf=\textbf\def\PY@tc##1{\textcolor[rgb]{0.73,0.40,0.13}{##1}}}
\expandafter\def\csname PY@tok@sr\endcsname{\def\PY@tc##1{\textcolor[rgb]{0.73,0.40,0.53}{##1}}}
\expandafter\def\csname PY@tok@ss\endcsname{\def\PY@tc##1{\textcolor[rgb]{0.10,0.09,0.49}{##1}}}
\expandafter\def\csname PY@tok@sx\endcsname{\def\PY@tc##1{\textcolor[rgb]{0.00,0.50,0.00}{##1}}}
\expandafter\def\csname PY@tok@m\endcsname{\def\PY@tc##1{\textcolor[rgb]{0.40,0.40,0.40}{##1}}}
\expandafter\def\csname PY@tok@gh\endcsname{\let\PY@bf=\textbf\def\PY@tc##1{\textcolor[rgb]{0.00,0.00,0.50}{##1}}}
\expandafter\def\csname PY@tok@gu\endcsname{\let\PY@bf=\textbf\def\PY@tc##1{\textcolor[rgb]{0.50,0.00,0.50}{##1}}}
\expandafter\def\csname PY@tok@gd\endcsname{\def\PY@tc##1{\textcolor[rgb]{0.63,0.00,0.00}{##1}}}
\expandafter\def\csname PY@tok@gi\endcsname{\def\PY@tc##1{\textcolor[rgb]{0.00,0.63,0.00}{##1}}}
\expandafter\def\csname PY@tok@gr\endcsname{\def\PY@tc##1{\textcolor[rgb]{1.00,0.00,0.00}{##1}}}
\expandafter\def\csname PY@tok@ge\endcsname{\let\PY@it=\textit}
\expandafter\def\csname PY@tok@gs\endcsname{\let\PY@bf=\textbf}
\expandafter\def\csname PY@tok@gp\endcsname{\let\PY@bf=\textbf\def\PY@tc##1{\textcolor[rgb]{0.00,0.00,0.50}{##1}}}
\expandafter\def\csname PY@tok@go\endcsname{\def\PY@tc##1{\textcolor[rgb]{0.53,0.53,0.53}{##1}}}
\expandafter\def\csname PY@tok@gt\endcsname{\def\PY@tc##1{\textcolor[rgb]{0.00,0.27,0.87}{##1}}}
\expandafter\def\csname PY@tok@err\endcsname{\def\PY@bc##1{\setlength{\fboxsep}{0pt}\fcolorbox[rgb]{1.00,0.00,0.00}{1,1,1}{\strut ##1}}}
\expandafter\def\csname PY@tok@kc\endcsname{\let\PY@bf=\textbf\def\PY@tc##1{\textcolor[rgb]{0.00,0.50,0.00}{##1}}}
\expandafter\def\csname PY@tok@kd\endcsname{\let\PY@bf=\textbf\def\PY@tc##1{\textcolor[rgb]{0.00,0.50,0.00}{##1}}}
\expandafter\def\csname PY@tok@kn\endcsname{\let\PY@bf=\textbf\def\PY@tc##1{\textcolor[rgb]{0.00,0.50,0.00}{##1}}}
\expandafter\def\csname PY@tok@kr\endcsname{\let\PY@bf=\textbf\def\PY@tc##1{\textcolor[rgb]{0.00,0.50,0.00}{##1}}}
\expandafter\def\csname PY@tok@bp\endcsname{\def\PY@tc##1{\textcolor[rgb]{0.00,0.50,0.00}{##1}}}
\expandafter\def\csname PY@tok@fm\endcsname{\def\PY@tc##1{\textcolor[rgb]{0.00,0.00,1.00}{##1}}}
\expandafter\def\csname PY@tok@vc\endcsname{\def\PY@tc##1{\textcolor[rgb]{0.10,0.09,0.49}{##1}}}
\expandafter\def\csname PY@tok@vg\endcsname{\def\PY@tc##1{\textcolor[rgb]{0.10,0.09,0.49}{##1}}}
\expandafter\def\csname PY@tok@vi\endcsname{\def\PY@tc##1{\textcolor[rgb]{0.10,0.09,0.49}{##1}}}
\expandafter\def\csname PY@tok@vm\endcsname{\def\PY@tc##1{\textcolor[rgb]{0.10,0.09,0.49}{##1}}}
\expandafter\def\csname PY@tok@sa\endcsname{\def\PY@tc##1{\textcolor[rgb]{0.73,0.13,0.13}{##1}}}
\expandafter\def\csname PY@tok@sb\endcsname{\def\PY@tc##1{\textcolor[rgb]{0.73,0.13,0.13}{##1}}}
\expandafter\def\csname PY@tok@sc\endcsname{\def\PY@tc##1{\textcolor[rgb]{0.73,0.13,0.13}{##1}}}
\expandafter\def\csname PY@tok@dl\endcsname{\def\PY@tc##1{\textcolor[rgb]{0.73,0.13,0.13}{##1}}}
\expandafter\def\csname PY@tok@s2\endcsname{\def\PY@tc##1{\textcolor[rgb]{0.73,0.13,0.13}{##1}}}
\expandafter\def\csname PY@tok@sh\endcsname{\def\PY@tc##1{\textcolor[rgb]{0.73,0.13,0.13}{##1}}}
\expandafter\def\csname PY@tok@s1\endcsname{\def\PY@tc##1{\textcolor[rgb]{0.73,0.13,0.13}{##1}}}
\expandafter\def\csname PY@tok@mb\endcsname{\def\PY@tc##1{\textcolor[rgb]{0.40,0.40,0.40}{##1}}}
\expandafter\def\csname PY@tok@mf\endcsname{\def\PY@tc##1{\textcolor[rgb]{0.40,0.40,0.40}{##1}}}
\expandafter\def\csname PY@tok@mh\endcsname{\def\PY@tc##1{\textcolor[rgb]{0.40,0.40,0.40}{##1}}}
\expandafter\def\csname PY@tok@mi\endcsname{\def\PY@tc##1{\textcolor[rgb]{0.40,0.40,0.40}{##1}}}
\expandafter\def\csname PY@tok@il\endcsname{\def\PY@tc##1{\textcolor[rgb]{0.40,0.40,0.40}{##1}}}
\expandafter\def\csname PY@tok@mo\endcsname{\def\PY@tc##1{\textcolor[rgb]{0.40,0.40,0.40}{##1}}}
\expandafter\def\csname PY@tok@ch\endcsname{\let\PY@it=\textit\def\PY@tc##1{\textcolor[rgb]{0.25,0.50,0.50}{##1}}}
\expandafter\def\csname PY@tok@cm\endcsname{\let\PY@it=\textit\def\PY@tc##1{\textcolor[rgb]{0.25,0.50,0.50}{##1}}}
\expandafter\def\csname PY@tok@cpf\endcsname{\let\PY@it=\textit\def\PY@tc##1{\textcolor[rgb]{0.25,0.50,0.50}{##1}}}
\expandafter\def\csname PY@tok@c1\endcsname{\let\PY@it=\textit\def\PY@tc##1{\textcolor[rgb]{0.25,0.50,0.50}{##1}}}
\expandafter\def\csname PY@tok@cs\endcsname{\let\PY@it=\textit\def\PY@tc##1{\textcolor[rgb]{0.25,0.50,0.50}{##1}}}

\def\PYZbs{\char`\\}
\def\PYZus{\char`\_}
\def\PYZob{\char`\{}
\def\PYZcb{\char`\}}
\def\PYZca{\char`\^}
\def\PYZam{\char`\&}
\def\PYZlt{\char`\<}
\def\PYZgt{\char`\>}
\def\PYZsh{\char`\#}
\def\PYZpc{\char`\%}
\def\PYZdl{\char`\$}
\def\PYZhy{\char`\-}
\def\PYZsq{\char`\'}
\def\PYZdq{\char`\"}
\def\PYZti{\char`\~}
% for compatibility with earlier versions
\def\PYZat{@}
\def\PYZlb{[}
\def\PYZrb{]}
\makeatother


    % Exact colors from NB
    \definecolor{incolor}{rgb}{0.0, 0.0, 0.5}
    \definecolor{outcolor}{rgb}{0.545, 0.0, 0.0}



    
    % Prevent overflowing lines due to hard-to-break entities
    \sloppy 
    % Setup hyperref package
    \hypersetup{
      breaklinks=true,  % so long urls are correctly broken across lines
      colorlinks=true,
      urlcolor=urlcolor,
      linkcolor=linkcolor,
      citecolor=citecolor,
      }
    % Slightly bigger margins than the latex defaults
    
    \geometry{verbose,tmargin=1in,bmargin=1in,lmargin=1in,rmargin=1in}
    
    

    \begin{document}
    
    
    \maketitle
    
    

    
    \hypertarget{ux5f00ux59cb}{%
\section{开始!}\label{ux5f00ux59cb}}

    \hypertarget{ux52a0ux8f7dux68c0ux67e5ux6a21ux578b}{%
\subsection{加载、检查模型}\label{ux52a0ux8f7dux68c0ux67e5ux6a21ux578b}}

    默认, cobrapy 为 \emph{沙门氏菌} 和 \emph{大肠杆菌} 有打包好的模型, 和
\emph{大肠杆菌} 核心代谢的一个 ``教科书'' 模型. 为了加载测试模型, 输入

    \begin{Verbatim}[commandchars=\\\{\}]
{\color{incolor}In [{\color{incolor}1}]:} \PY{k+kn}{from} \PY{n+nn}{\PYZus{}\PYZus{}future\PYZus{}\PYZus{}} \PY{k}{import} \PY{n}{print\PYZus{}function}
        
        \PY{k+kn}{import} \PY{n+nn}{cobra}
        \PY{k+kn}{import} \PY{n+nn}{cobra}\PY{n+nn}{.}\PY{n+nn}{test}
        
        \PY{c+c1}{\PYZsh{} \PYZdq{}ecoli\PYZdq{} and \PYZdq{}salmonella\PYZdq{} are also valid arguments}
        \PY{n}{model} \PY{o}{=} \PY{n}{cobra}\PY{o}{.}\PY{n}{test}\PY{o}{.}\PY{n}{create\PYZus{}test\PYZus{}model}\PY{p}{(}\PY{l+s+s2}{\PYZdq{}}\PY{l+s+s2}{textbook}\PY{l+s+s2}{\PYZdq{}}\PY{p}{)}
\end{Verbatim}


    cobrapy 模型的反应, 代谢物, 和基因属性是一种特殊的被称为
\texttt{cobra.DictList} 的列表, 列表中的每一项由
\texttt{cobra.Reaction}, \texttt{cobra.Metabolite} and
\texttt{cobra.Gene} 对象分别组成.

    \begin{Verbatim}[commandchars=\\\{\}]
{\color{incolor}In [{\color{incolor}2}]:} \PY{n+nb}{print}\PY{p}{(}\PY{n+nb}{len}\PY{p}{(}\PY{n}{model}\PY{o}{.}\PY{n}{reactions}\PY{p}{)}\PY{p}{)}
        \PY{n+nb}{print}\PY{p}{(}\PY{n+nb}{len}\PY{p}{(}\PY{n}{model}\PY{o}{.}\PY{n}{metabolites}\PY{p}{)}\PY{p}{)}
        \PY{n+nb}{print}\PY{p}{(}\PY{n+nb}{len}\PY{p}{(}\PY{n}{model}\PY{o}{.}\PY{n}{genes}\PY{p}{)}\PY{p}{)}
\end{Verbatim}


    \begin{Verbatim}[commandchars=\\\{\}]
95
72
137

    \end{Verbatim}

    使用
\href{https://jupyter-notebook-beginner-guide.readthedocs.io/en/latest/}{Jupyter
notebook} 时这类信息显示为列表。

    \begin{Verbatim}[commandchars=\\\{\}]
{\color{incolor}In [{\color{incolor}3}]:} \PY{n}{model}
\end{Verbatim}


\begin{Verbatim}[commandchars=\\\{\}]
{\color{outcolor}Out[{\color{outcolor}3}]:} <Model e\_coli\_core at 0x1116ea9e8>
\end{Verbatim}
            
    像一个普通列表一样, \texttt{DictList} 中的对象可用 index 取得. 比如说,
在模型中取第 30 个反应 (使用了
\href{https://en.wikipedia.org/wiki/Zero-based_numbering}{0-indexing},所以
index 是 29):

    \begin{Verbatim}[commandchars=\\\{\}]
{\color{incolor}In [{\color{incolor}4}]:} \PY{n}{model}\PY{o}{.}\PY{n}{reactions}\PY{p}{[}\PY{l+m+mi}{29}\PY{p}{]}
\end{Verbatim}


\begin{Verbatim}[commandchars=\\\{\}]
{\color{outcolor}Out[{\color{outcolor}4}]:} <Reaction EX\_glu\_\_L\_e at 0x11b8643c8>
\end{Verbatim}
            
    额外地, 元素可以由它们的 \texttt{id},使用
\texttt{DictList.get\_by\_id()} 函数取得. 比如说,为了取得细胞溶质 ATP
代谢物 (cytosolic atp metabolite,id 是 ``atp\_c''), 我们可以这样做:

    \begin{Verbatim}[commandchars=\\\{\}]
{\color{incolor}In [{\color{incolor}5}]:} \PY{n}{model}\PY{o}{.}\PY{n}{metabolites}\PY{o}{.}\PY{n}{get\PYZus{}by\PYZus{}id}\PY{p}{(}\PY{l+s+s2}{\PYZdq{}}\PY{l+s+s2}{atp\PYZus{}c}\PY{l+s+s2}{\PYZdq{}}\PY{p}{)}
\end{Verbatim}


\begin{Verbatim}[commandchars=\\\{\}]
{\color{outcolor}Out[{\color{outcolor}5}]:} <Metabolite atp\_c at 0x11b7f82b0>
\end{Verbatim}
            
    额外的一个加成是, 使用交互式 shell,诸如 IPython 的用户可以用 tab
补全来在列表中列举元素. 虽然对大部分的代码我们不太建议这么做,因为 id
中可能会有 ``-'' 这样的字符, 在交互式 prompt 中这很有用:

    \begin{Verbatim}[commandchars=\\\{\}]
{\color{incolor}In [{\color{incolor}6}]:} \PY{n}{model}\PY{o}{.}\PY{n}{reactions}\PY{o}{.}\PY{n}{EX\PYZus{}glc\PYZus{}\PYZus{}D\PYZus{}e}\PY{o}{.}\PY{n}{bounds}
\end{Verbatim}


\begin{Verbatim}[commandchars=\\\{\}]
{\color{outcolor}Out[{\color{outcolor}6}]:} (-10.0, 1000.0)
\end{Verbatim}
            
    \hypertarget{ux53cdux5e94}{%
\subsection{反应}\label{ux53cdux5e94}}

    我们将考虑葡萄糖-6-磷酸异构酶 (glucose 6-phosphate isomerase) 反应,
它相互转换了葡萄糖6-磷酸 (glucose 6-phosphate) 和果糖6-磷酸 (fructose
6-phosphate). 这个反应的 id 在我们的测试模型中是 PGI.

    \begin{Verbatim}[commandchars=\\\{\}]
{\color{incolor}In [{\color{incolor}7}]:} \PY{n}{pgi} \PY{o}{=} \PY{n}{model}\PY{o}{.}\PY{n}{reactions}\PY{o}{.}\PY{n}{get\PYZus{}by\PYZus{}id}\PY{p}{(}\PY{l+s+s2}{\PYZdq{}}\PY{l+s+s2}{PGI}\PY{l+s+s2}{\PYZdq{}}\PY{p}{)}
        \PY{n}{pgi}
\end{Verbatim}


\begin{Verbatim}[commandchars=\\\{\}]
{\color{outcolor}Out[{\color{outcolor}7}]:} <Reaction PGI at 0x11b886a90>
\end{Verbatim}
            
    我们可以看到全名和催化(?)为字符串的反应。

    \begin{Verbatim}[commandchars=\\\{\}]
{\color{incolor}In [{\color{incolor}8}]:} \PY{n+nb}{print}\PY{p}{(}\PY{n}{pgi}\PY{o}{.}\PY{n}{name}\PY{p}{)}
        \PY{n+nb}{print}\PY{p}{(}\PY{n}{pgi}\PY{o}{.}\PY{n}{reaction}\PY{p}{)}
\end{Verbatim}


    \begin{Verbatim}[commandchars=\\\{\}]
glucose-6-phosphate isomerase
g6p\_c <=> f6p\_c

    \end{Verbatim}

    我们也可以看到反应的上下限. 因为 \texttt{pgi.lower\_bound} \textless{}
0, 而且 \texttt{pgi.upper\_bound} \textgreater{} 0, \texttt{pgi}
是可逆的.

    \begin{Verbatim}[commandchars=\\\{\}]
{\color{incolor}In [{\color{incolor}9}]:} \PY{n+nb}{print}\PY{p}{(}\PY{n}{pgi}\PY{o}{.}\PY{n}{lower\PYZus{}bound}\PY{p}{,} \PY{l+s+s2}{\PYZdq{}}\PY{l+s+s2}{\PYZlt{} pgi \PYZlt{}}\PY{l+s+s2}{\PYZdq{}}\PY{p}{,} \PY{n}{pgi}\PY{o}{.}\PY{n}{upper\PYZus{}bound}\PY{p}{)}
        \PY{n+nb}{print}\PY{p}{(}\PY{n}{pgi}\PY{o}{.}\PY{n}{reversibility}\PY{p}{)}
\end{Verbatim}


    \begin{Verbatim}[commandchars=\\\{\}]
-1000.0 < pgi < 1000.0
True

    \end{Verbatim}

    我们也可以保证这个反应已经质量配平. 这个函数会返回与质量守恒相悖的元素.
如果它返回空,那么反应已配平.

    \begin{Verbatim}[commandchars=\\\{\}]
{\color{incolor}In [{\color{incolor}10}]:} \PY{n}{pgi}\PY{o}{.}\PY{n}{check\PYZus{}mass\PYZus{}balance}\PY{p}{(}\PY{p}{)}
\end{Verbatim}


\begin{Verbatim}[commandchars=\\\{\}]
{\color{outcolor}Out[{\color{outcolor}10}]:} \{\}
\end{Verbatim}
            
    为了添加反应物, 我们传递一个 \texttt{dict},带一个反应物对象和它的系数。

    \begin{Verbatim}[commandchars=\\\{\}]
{\color{incolor}In [{\color{incolor}11}]:} \PY{n}{pgi}\PY{o}{.}\PY{n}{add\PYZus{}metabolites}\PY{p}{(}\PY{p}{\PYZob{}}\PY{n}{model}\PY{o}{.}\PY{n}{metabolites}\PY{o}{.}\PY{n}{get\PYZus{}by\PYZus{}id}\PY{p}{(}\PY{l+s+s2}{\PYZdq{}}\PY{l+s+s2}{h\PYZus{}c}\PY{l+s+s2}{\PYZdq{}}\PY{p}{)}\PY{p}{:} \PY{o}{\PYZhy{}}\PY{l+m+mi}{1}\PY{p}{\PYZcb{}}\PY{p}{)}
         \PY{n}{pgi}\PY{o}{.}\PY{n}{reaction}
\end{Verbatim}


\begin{Verbatim}[commandchars=\\\{\}]
{\color{outcolor}Out[{\color{outcolor}11}]:} 'g6p\_c + h\_c <=> f6p\_c'
\end{Verbatim}
            
    这个反应不再配平了

    \begin{Verbatim}[commandchars=\\\{\}]
{\color{incolor}In [{\color{incolor}11}]:} \PY{n}{pgi}\PY{o}{.}\PY{n}{check\PYZus{}mass\PYZus{}balance}\PY{p}{(}\PY{p}{)}
\end{Verbatim}


\begin{Verbatim}[commandchars=\\\{\}]
{\color{outcolor}Out[{\color{outcolor}11}]:} \{'H': -1.0, 'charge': -1.0\}
\end{Verbatim}
            
    我们可以去掉这个反应物, 于是它又质量守恒啦.

    \begin{Verbatim}[commandchars=\\\{\}]
{\color{incolor}In [{\color{incolor}12}]:} \PY{n}{pgi}\PY{o}{.}\PY{n}{subtract\PYZus{}metabolites}\PY{p}{(}\PY{p}{\PYZob{}}\PY{n}{model}\PY{o}{.}\PY{n}{metabolites}\PY{o}{.}\PY{n}{get\PYZus{}by\PYZus{}id}\PY{p}{(}\PY{l+s+s2}{\PYZdq{}}\PY{l+s+s2}{h\PYZus{}c}\PY{l+s+s2}{\PYZdq{}}\PY{p}{)}\PY{p}{:} \PY{o}{\PYZhy{}}\PY{l+m+mi}{1}\PY{p}{\PYZcb{}}\PY{p}{)}
         \PY{n+nb}{print}\PY{p}{(}\PY{n}{pgi}\PY{o}{.}\PY{n}{reaction}\PY{p}{)}
         \PY{n+nb}{print}\PY{p}{(}\PY{n}{pgi}\PY{o}{.}\PY{n}{check\PYZus{}mass\PYZus{}balance}\PY{p}{(}\PY{p}{)}\PY{p}{)}
\end{Verbatim}


    \begin{Verbatim}[commandchars=\\\{\}]
g6p\_c <=> f6p\_c
\{\}

    \end{Verbatim}

    我们也可以从字符串构建反应. 但是, 一定要小心,保证反应中的 id
和模型中的那些相符. 箭头的方向也被使用来更新上下限.

    \begin{Verbatim}[commandchars=\\\{\}]
{\color{incolor}In [{\color{incolor}13}]:} \PY{n}{pgi}\PY{o}{.}\PY{n}{reaction} \PY{o}{=} \PY{l+s+s2}{\PYZdq{}}\PY{l+s+s2}{g6p\PYZus{}c \PYZhy{}\PYZhy{}\PYZgt{} f6p\PYZus{}c + h\PYZus{}c + green\PYZus{}eggs + ham}\PY{l+s+s2}{\PYZdq{}}
\end{Verbatim}


    \begin{Verbatim}[commandchars=\\\{\}]
unknown metabolite 'green\_eggs' created
unknown metabolite 'ham' created

    \end{Verbatim}

    \begin{Verbatim}[commandchars=\\\{\}]
{\color{incolor}In [{\color{incolor}14}]:} \PY{n}{pgi}\PY{o}{.}\PY{n}{reaction}
\end{Verbatim}


\begin{Verbatim}[commandchars=\\\{\}]
{\color{outcolor}Out[{\color{outcolor}14}]:} 'g6p\_c --> f6p\_c + green\_eggs + h\_c + ham'
\end{Verbatim}
            
    \begin{Verbatim}[commandchars=\\\{\}]
{\color{incolor}In [{\color{incolor}15}]:} \PY{n}{pgi}\PY{o}{.}\PY{n}{reaction} \PY{o}{=} \PY{l+s+s2}{\PYZdq{}}\PY{l+s+s2}{g6p\PYZus{}c \PYZlt{}=\PYZgt{} f6p\PYZus{}c}\PY{l+s+s2}{\PYZdq{}}
         \PY{n}{pgi}\PY{o}{.}\PY{n}{reaction}
\end{Verbatim}


\begin{Verbatim}[commandchars=\\\{\}]
{\color{outcolor}Out[{\color{outcolor}15}]:} 'g6p\_c <=> f6p\_c'
\end{Verbatim}
            
    \hypertarget{ux4ee3ux8c22ux7269}{%
\subsection{代谢物}\label{ux4ee3ux8c22ux7269}}

    我们将细胞溶质 ATP (cytosolic atp) 作为我们的代谢物,
它在我们的测试模型中的 id 是 \texttt{"atp\_c"}.

    \begin{Verbatim}[commandchars=\\\{\}]
{\color{incolor}In [{\color{incolor}16}]:} \PY{n}{atp} \PY{o}{=} \PY{n}{model}\PY{o}{.}\PY{n}{metabolites}\PY{o}{.}\PY{n}{get\PYZus{}by\PYZus{}id}\PY{p}{(}\PY{l+s+s2}{\PYZdq{}}\PY{l+s+s2}{atp\PYZus{}c}\PY{l+s+s2}{\PYZdq{}}\PY{p}{)}
         \PY{n}{atp}
\end{Verbatim}


\begin{Verbatim}[commandchars=\\\{\}]
{\color{outcolor}Out[{\color{outcolor}16}]:} <Metabolite atp\_c at 0x11b7f82b0>
\end{Verbatim}
            
    我们可以直接以字符串打印代谢物名称和 compartment (在这里是胞质溶胶
(cytosol)).

    \begin{Verbatim}[commandchars=\\\{\}]
{\color{incolor}In [{\color{incolor}17}]:} \PY{n+nb}{print}\PY{p}{(}\PY{n}{atp}\PY{o}{.}\PY{n}{name}\PY{p}{)}
         \PY{n+nb}{print}\PY{p}{(}\PY{n}{atp}\PY{o}{.}\PY{n}{compartment}\PY{p}{)}
\end{Verbatim}


    \begin{Verbatim}[commandchars=\\\{\}]
ATP
c

    \end{Verbatim}

    我们可以看到 ATP 在我们的模型是是一个带电分子。

    \begin{Verbatim}[commandchars=\\\{\}]
{\color{incolor}In [{\color{incolor}18}]:} \PY{n}{atp}\PY{o}{.}\PY{n}{charge}
\end{Verbatim}


\begin{Verbatim}[commandchars=\\\{\}]
{\color{outcolor}Out[{\color{outcolor}18}]:} -4
\end{Verbatim}
            
    我们也可以看到代谢物的化学方程式

    \begin{Verbatim}[commandchars=\\\{\}]
{\color{incolor}In [{\color{incolor}19}]:} \PY{n+nb}{print}\PY{p}{(}\PY{n}{atp}\PY{o}{.}\PY{n}{formula}\PY{p}{)}
\end{Verbatim}


    \begin{Verbatim}[commandchars=\\\{\}]
C10H12N5O13P3

    \end{Verbatim}

    反应的属性提供了一个所有使用当前代谢物的反应的 \texttt{frozenset}.
我们可以用它来统计所有使用 atp 的反应.

    \begin{Verbatim}[commandchars=\\\{\}]
{\color{incolor}In [{\color{incolor}20}]:} \PY{n+nb}{len}\PY{p}{(}\PY{n}{atp}\PY{o}{.}\PY{n}{reactions}\PY{p}{)}
\end{Verbatim}


\begin{Verbatim}[commandchars=\\\{\}]
{\color{outcolor}Out[{\color{outcolor}20}]:} 13
\end{Verbatim}
            
    像葡萄糖6-磷酸 (glucose 6-phosphate) 参与的反应就少一点.

    \begin{Verbatim}[commandchars=\\\{\}]
{\color{incolor}In [{\color{incolor}21}]:} \PY{n}{model}\PY{o}{.}\PY{n}{metabolites}\PY{o}{.}\PY{n}{get\PYZus{}by\PYZus{}id}\PY{p}{(}\PY{l+s+s2}{\PYZdq{}}\PY{l+s+s2}{g6p\PYZus{}c}\PY{l+s+s2}{\PYZdq{}}\PY{p}{)}\PY{o}{.}\PY{n}{reactions}
\end{Verbatim}


\begin{Verbatim}[commandchars=\\\{\}]
{\color{outcolor}Out[{\color{outcolor}21}]:} frozenset(\{<Reaction G6PDH2r at 0x11b870c88>,
                    <Reaction GLCpts at 0x11b870f98>,
                    <Reaction PGI at 0x11b886a90>,
                    <Reaction Biomass\_Ecoli\_core at 0x11b85a5f8>\})
\end{Verbatim}
            
    \hypertarget{ux57faux56e0}{%
\subsection{基因}\label{ux57faux56e0}}

    \texttt{gene\_reaction\_rule} 是一个反应激活的基因要求的布尔表示,在
\href{http://dx.doi.org/doi:10.1038/nprot.2011.308}{Schellenberger et al
2011 Nature Protocols 6(9):1290-307} 中描述.

GPR 为作为字符串的反应对象存储为 gene\_reaction\_rule.

    \begin{Verbatim}[commandchars=\\\{\}]
{\color{incolor}In [{\color{incolor}22}]:} \PY{n}{gpr} \PY{o}{=} \PY{n}{pgi}\PY{o}{.}\PY{n}{gene\PYZus{}reaction\PYZus{}rule}
         \PY{n}{gpr}
\end{Verbatim}


\begin{Verbatim}[commandchars=\\\{\}]
{\color{outcolor}Out[{\color{outcolor}22}]:} 'b4025'
\end{Verbatim}
            
    相关的基因对象也存在. 这些对象由反应和模型追踪。

    \begin{Verbatim}[commandchars=\\\{\}]
{\color{incolor}In [{\color{incolor}23}]:} \PY{n}{pgi}\PY{o}{.}\PY{n}{genes}
\end{Verbatim}


\begin{Verbatim}[commandchars=\\\{\}]
{\color{outcolor}Out[{\color{outcolor}23}]:} frozenset(\{<Gene b4025 at 0x11b844cc0>\})
\end{Verbatim}
            
    \begin{Verbatim}[commandchars=\\\{\}]
{\color{incolor}In [{\color{incolor}24}]:} \PY{n}{pgi\PYZus{}gene} \PY{o}{=} \PY{n}{model}\PY{o}{.}\PY{n}{genes}\PY{o}{.}\PY{n}{get\PYZus{}by\PYZus{}id}\PY{p}{(}\PY{l+s+s2}{\PYZdq{}}\PY{l+s+s2}{b4025}\PY{l+s+s2}{\PYZdq{}}\PY{p}{)}
         \PY{n}{pgi\PYZus{}gene}
\end{Verbatim}


\begin{Verbatim}[commandchars=\\\{\}]
{\color{outcolor}Out[{\color{outcolor}24}]:} <Gene b4025 at 0x11b844cc0>
\end{Verbatim}
            
    每个基因记录了它催化的反应对象。

    \begin{Verbatim}[commandchars=\\\{\}]
{\color{incolor}In [{\color{incolor}25}]:} \PY{n}{pgi\PYZus{}gene}\PY{o}{.}\PY{n}{reactions}
\end{Verbatim}


\begin{Verbatim}[commandchars=\\\{\}]
{\color{outcolor}Out[{\color{outcolor}25}]:} frozenset(\{<Reaction PGI at 0x11b886a90>\})
\end{Verbatim}
            
    在必要时,修改 gene\_reaction\_rule
会创建新的基因对象,并且更新所有关系.

    \begin{Verbatim}[commandchars=\\\{\}]
{\color{incolor}In [{\color{incolor}26}]:} \PY{n}{pgi}\PY{o}{.}\PY{n}{gene\PYZus{}reaction\PYZus{}rule} \PY{o}{=} \PY{l+s+s2}{\PYZdq{}}\PY{l+s+s2}{(spam or eggs)}\PY{l+s+s2}{\PYZdq{}}
         \PY{n}{pgi}\PY{o}{.}\PY{n}{genes}
\end{Verbatim}


\begin{Verbatim}[commandchars=\\\{\}]
{\color{outcolor}Out[{\color{outcolor}26}]:} frozenset(\{<Gene spam at 0x11b850908>, <Gene eggs at 0x11b850eb8>\})
\end{Verbatim}
            
    \begin{Verbatim}[commandchars=\\\{\}]
{\color{incolor}In [{\color{incolor}27}]:} \PY{n}{pgi\PYZus{}gene}\PY{o}{.}\PY{n}{reactions}
\end{Verbatim}


\begin{Verbatim}[commandchars=\\\{\}]
{\color{outcolor}Out[{\color{outcolor}27}]:} frozenset()
\end{Verbatim}
            
    新建的基因也被加入模型中

    \begin{Verbatim}[commandchars=\\\{\}]
{\color{incolor}In [{\color{incolor}28}]:} \PY{n}{model}\PY{o}{.}\PY{n}{genes}\PY{o}{.}\PY{n}{get\PYZus{}by\PYZus{}id}\PY{p}{(}\PY{l+s+s2}{\PYZdq{}}\PY{l+s+s2}{spam}\PY{l+s+s2}{\PYZdq{}}\PY{p}{)}
\end{Verbatim}


\begin{Verbatim}[commandchars=\\\{\}]
{\color{outcolor}Out[{\color{outcolor}28}]:} <Gene spam at 0x11b850908>
\end{Verbatim}
            
    \texttt{delete\_model\_genes} 函数会估计
GPR,如果反应被敲除的话,它的上下限都会设置为 0. 使用
\texttt{cumulative\_deletions}
flag,这个函数可以存储已有的删除操作或者重置它们.

    \begin{Verbatim}[commandchars=\\\{\}]
{\color{incolor}In [{\color{incolor}29}]:} \PY{n}{cobra}\PY{o}{.}\PY{n}{manipulation}\PY{o}{.}\PY{n}{delete\PYZus{}model\PYZus{}genes}\PY{p}{(}
             \PY{n}{model}\PY{p}{,} \PY{p}{[}\PY{l+s+s2}{\PYZdq{}}\PY{l+s+s2}{spam}\PY{l+s+s2}{\PYZdq{}}\PY{p}{]}\PY{p}{,} \PY{n}{cumulative\PYZus{}deletions}\PY{o}{=}\PY{k+kc}{True}\PY{p}{)}
         \PY{n+nb}{print}\PY{p}{(}\PY{l+s+s2}{\PYZdq{}}\PY{l+s+s2}{after 1 KO: }\PY{l+s+si}{\PYZpc{}4d}\PY{l+s+s2}{ \PYZlt{} flux\PYZus{}PGI \PYZlt{} }\PY{l+s+si}{\PYZpc{}4d}\PY{l+s+s2}{\PYZdq{}} \PY{o}{\PYZpc{}} \PY{p}{(}\PY{n}{pgi}\PY{o}{.}\PY{n}{lower\PYZus{}bound}\PY{p}{,} \PY{n}{pgi}\PY{o}{.}\PY{n}{upper\PYZus{}bound}\PY{p}{)}\PY{p}{)}
         
         \PY{n}{cobra}\PY{o}{.}\PY{n}{manipulation}\PY{o}{.}\PY{n}{delete\PYZus{}model\PYZus{}genes}\PY{p}{(}
             \PY{n}{model}\PY{p}{,} \PY{p}{[}\PY{l+s+s2}{\PYZdq{}}\PY{l+s+s2}{eggs}\PY{l+s+s2}{\PYZdq{}}\PY{p}{]}\PY{p}{,} \PY{n}{cumulative\PYZus{}deletions}\PY{o}{=}\PY{k+kc}{True}\PY{p}{)}
         \PY{n+nb}{print}\PY{p}{(}\PY{l+s+s2}{\PYZdq{}}\PY{l+s+s2}{after 2 KO:  }\PY{l+s+si}{\PYZpc{}4d}\PY{l+s+s2}{ \PYZlt{} flux\PYZus{}PGI \PYZlt{} }\PY{l+s+si}{\PYZpc{}4d}\PY{l+s+s2}{\PYZdq{}} \PY{o}{\PYZpc{}} \PY{p}{(}\PY{n}{pgi}\PY{o}{.}\PY{n}{lower\PYZus{}bound}\PY{p}{,} \PY{n}{pgi}\PY{o}{.}\PY{n}{upper\PYZus{}bound}\PY{p}{)}\PY{p}{)}
\end{Verbatim}


    \begin{Verbatim}[commandchars=\\\{\}]
after 1 KO: -1000 < flux\_PGI < 1000
after 2 KO:     0 < flux\_PGI <    0

    \end{Verbatim}

    undelete\_model\_genes 能用来重置基因删除操作

    \begin{Verbatim}[commandchars=\\\{\}]
{\color{incolor}In [{\color{incolor}30}]:} \PY{n}{cobra}\PY{o}{.}\PY{n}{manipulation}\PY{o}{.}\PY{n}{undelete\PYZus{}model\PYZus{}genes}\PY{p}{(}\PY{n}{model}\PY{p}{)}
         \PY{n+nb}{print}\PY{p}{(}\PY{n}{pgi}\PY{o}{.}\PY{n}{lower\PYZus{}bound}\PY{p}{,} \PY{l+s+s2}{\PYZdq{}}\PY{l+s+s2}{\PYZlt{} pgi \PYZlt{}}\PY{l+s+s2}{\PYZdq{}}\PY{p}{,} \PY{n}{pgi}\PY{o}{.}\PY{n}{upper\PYZus{}bound}\PY{p}{)}
\end{Verbatim}


    \begin{Verbatim}[commandchars=\\\{\}]
-1000 < pgi < 1000

    \end{Verbatim}

    \hypertarget{ux901aux8fc7ux6a21ux578bux4f5cux4e3aux4e0aux4e0bux6587ux4f7fux6539ux53d8ux53efux56deux6eaf}{%
\subsection{通过模型作为上下文使改变可回溯}\label{ux901aux8fc7ux6a21ux578bux4f5cux4e3aux4e0aux4e0bux6587ux4f7fux6539ux53d8ux53efux56deux6eaf}}

    经常, 有人会想对模型细微改变然后评估影响. 比如,
我们可能想依序敲除所有反应, 然后看它们会对目标函数 (objective function)
带来什么改变. 一种方式是在敲除之前用 \texttt{model.copy()}
创建模型的一个新副本. 但是, 即使是小模型,
这也会很慢,因为模型是非常复杂的对象.
更好的办法是在处理下一次反应前进行敲除、优化和手动重置反应限度.
既然这种情况经常发生, cobrapy 允许我们使用模型作为上下文,
来自动回退改变.

    \begin{Verbatim}[commandchars=\\\{\}]
{\color{incolor}In [{\color{incolor}31}]:} \PY{n}{model} \PY{o}{=} \PY{n}{cobra}\PY{o}{.}\PY{n}{test}\PY{o}{.}\PY{n}{create\PYZus{}test\PYZus{}model}\PY{p}{(}\PY{l+s+s1}{\PYZsq{}}\PY{l+s+s1}{textbook}\PY{l+s+s1}{\PYZsq{}}\PY{p}{)}
         \PY{k}{for} \PY{n}{reaction} \PY{o+ow}{in} \PY{n}{model}\PY{o}{.}\PY{n}{reactions}\PY{p}{[}\PY{p}{:}\PY{l+m+mi}{5}\PY{p}{]}\PY{p}{:}
             \PY{k}{with} \PY{n}{model} \PY{k}{as} \PY{n}{model}\PY{p}{:}
                 \PY{n}{reaction}\PY{o}{.}\PY{n}{knock\PYZus{}out}\PY{p}{(}\PY{p}{)}
                 \PY{n}{model}\PY{o}{.}\PY{n}{optimize}\PY{p}{(}\PY{p}{)}
                 \PY{n+nb}{print}\PY{p}{(}\PY{l+s+s1}{\PYZsq{}}\PY{l+s+si}{\PYZpc{}s}\PY{l+s+s1}{ blocked (bounds: }\PY{l+s+si}{\PYZpc{}s}\PY{l+s+s1}{), new growth rate }\PY{l+s+si}{\PYZpc{}f}\PY{l+s+s1}{\PYZsq{}} \PY{o}{\PYZpc{}}
                       \PY{p}{(}\PY{n}{reaction}\PY{o}{.}\PY{n}{id}\PY{p}{,} \PY{n+nb}{str}\PY{p}{(}\PY{n}{reaction}\PY{o}{.}\PY{n}{bounds}\PY{p}{)}\PY{p}{,} \PY{n}{model}\PY{o}{.}\PY{n}{objective}\PY{o}{.}\PY{n}{value}\PY{p}{)}\PY{p}{)}
\end{Verbatim}


    \begin{Verbatim}[commandchars=\\\{\}]
ACALD blocked (bounds: (0, 0)), new growth rate 0.873922
ACALDt blocked (bounds: (0, 0)), new growth rate 0.873922
ACKr blocked (bounds: (0, 0)), new growth rate 0.873922
ACONTa blocked (bounds: (0, 0)), new growth rate -0.000000
ACONTb blocked (bounds: (0, 0)), new growth rate -0.000000

    \end{Verbatim}

    如果我们看一下被敲除的反应, 可以看到它们的上下限都被回退到原值了.

    \begin{Verbatim}[commandchars=\\\{\}]
{\color{incolor}In [{\color{incolor}32}]:} \PY{p}{[}\PY{n}{reaction}\PY{o}{.}\PY{n}{bounds} \PY{k}{for} \PY{n}{reaction} \PY{o+ow}{in} \PY{n}{model}\PY{o}{.}\PY{n}{reactions}\PY{p}{[}\PY{p}{:}\PY{l+m+mi}{5}\PY{p}{]}\PY{p}{]}
\end{Verbatim}


\begin{Verbatim}[commandchars=\\\{\}]
{\color{outcolor}Out[{\color{outcolor}32}]:} [(-1000.0, 1000.0),
          (-1000.0, 1000.0),
          (-1000.0, 1000.0),
          (-1000.0, 1000.0),
          (-1000.0, 1000.0)]
\end{Verbatim}
            
    嵌套上下文也是支持的。

    \begin{Verbatim}[commandchars=\\\{\}]
{\color{incolor}In [{\color{incolor}33}]:} \PY{n+nb}{print}\PY{p}{(}\PY{l+s+s1}{\PYZsq{}}\PY{l+s+s1}{original objective: }\PY{l+s+s1}{\PYZsq{}}\PY{p}{,} \PY{n}{model}\PY{o}{.}\PY{n}{objective}\PY{o}{.}\PY{n}{expression}\PY{p}{)}
         \PY{k}{with} \PY{n}{model}\PY{p}{:}
             \PY{n}{model}\PY{o}{.}\PY{n}{objective} \PY{o}{=} \PY{l+s+s1}{\PYZsq{}}\PY{l+s+s1}{ATPM}\PY{l+s+s1}{\PYZsq{}}
             \PY{n+nb}{print}\PY{p}{(}\PY{l+s+s1}{\PYZsq{}}\PY{l+s+s1}{print objective in first context:}\PY{l+s+s1}{\PYZsq{}}\PY{p}{,} \PY{n}{model}\PY{o}{.}\PY{n}{objective}\PY{o}{.}\PY{n}{expression}\PY{p}{)}
             \PY{k}{with} \PY{n}{model}\PY{p}{:}
                 \PY{n}{model}\PY{o}{.}\PY{n}{objective} \PY{o}{=} \PY{l+s+s1}{\PYZsq{}}\PY{l+s+s1}{ACALD}\PY{l+s+s1}{\PYZsq{}}
                 \PY{n+nb}{print}\PY{p}{(}\PY{l+s+s1}{\PYZsq{}}\PY{l+s+s1}{print objective in second context:}\PY{l+s+s1}{\PYZsq{}}\PY{p}{,} \PY{n}{model}\PY{o}{.}\PY{n}{objective}\PY{o}{.}\PY{n}{expression}\PY{p}{)}
             \PY{n+nb}{print}\PY{p}{(}\PY{l+s+s1}{\PYZsq{}}\PY{l+s+s1}{objective after exiting second context:}\PY{l+s+s1}{\PYZsq{}}\PY{p}{,}
                   \PY{n}{model}\PY{o}{.}\PY{n}{objective}\PY{o}{.}\PY{n}{expression}\PY{p}{)}
         \PY{n+nb}{print}\PY{p}{(}\PY{l+s+s1}{\PYZsq{}}\PY{l+s+s1}{back to original objective:}\PY{l+s+s1}{\PYZsq{}}\PY{p}{,} \PY{n}{model}\PY{o}{.}\PY{n}{objective}\PY{o}{.}\PY{n}{expression}\PY{p}{)}
\end{Verbatim}


    \begin{Verbatim}[commandchars=\\\{\}]
original objective:  -1.0*Biomass\_Ecoli\_core\_reverse\_2cdba + 1.0*Biomass\_Ecoli\_core
print objective in first context: -1.0*ATPM\_reverse\_5b752 + 1.0*ATPM
print objective in second context: 1.0*ACALD - 1.0*ACALD\_reverse\_fda2b
objective after exiting second context: -1.0*ATPM\_reverse\_5b752 + 1.0*ATPM
back to original objective: -1.0*Biomass\_Ecoli\_core\_reverse\_2cdba + 1.0*Biomass\_Ecoli\_core

    \end{Verbatim}

    大部分会改变模型的方法,包括添加、删除反应和代谢物、设置目标,都支持这样做。支持的方法和函数会在相关文档中说明。

    虽然不会有实际影响,
为了语法方便,也可以在上下文外用不同的名字指向模型,比如说

    \begin{Verbatim}[commandchars=\\\{\}]
{\color{incolor}In [{\color{incolor}34}]:} \PY{k}{with} \PY{n}{model} \PY{k}{as} \PY{n}{inner}\PY{p}{:}
             \PY{n}{inner}\PY{o}{.}\PY{n}{reactions}\PY{o}{.}\PY{n}{PFK}\PY{o}{.}\PY{n}{knock\PYZus{}out}
\end{Verbatim}



    % Add a bibliography block to the postdoc
    
    
    
    \end{document}
